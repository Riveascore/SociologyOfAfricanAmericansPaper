\documentclass{sig-alternate}
\usepackage{cite}

\begin{document}

\title{Community's effect on mental health and differences by race of perceived sense of community.}
\author{
\alignauthor
Chris Aga \\
\affaddr{ 
University of Minnesota, Morris \\
600 East 4th St. \\
Morris, Minnesota 56267} \\
\email{agaxx010@morris.umn.edu}
}

\date{Today}

\maketitle


\begin{abstract}
Abstract.
\end{abstract}

\keywords{}

\section{Introduction}
Intro.


\section{Background}

%%%%%%%%%%%%%%%%%%%%%%%
Sense of Community as defined by McMillan in~\cite{senseOfCommunity:1996} as ``a spirit of belonging together, a feeling that there is an authority structure that can be trusted, an awareness that trade, and mutual benefit come from being together, and a spirit that comes from shared experiences that are preseved as art''.
%%%%%%%%%%%%%%%%%%%%%%%




%%%%%%%%%%%%%%%%%%%%%%%
``sense of
community is setting specific.''~\cite{cognitiveLearning:2002}
no universally accepted def
%%%%%%%%%%%%%%%%%%%%%%%

%%%%%%%%%%%%%%%%
Thus SOC is what is known as a latent construct, or as Garger explains in~\cite{latent:2011}, variables that cannot be directly observed ``such as extraversion, intelligence, and self-image''.
%.................

%%%%%%%%%%%%%%%%%%%%%%%
sense of community index~\cite{disparities:2009}
Four dimensions -> membership/influence/reinforcementOfNeeds/sharedEmotionalConnection

Differential Item Functioning to determine accuracy of SOC differences

``DIF among racial groups'' had not been previously studied

DIF applied to -> age/gender/maritialStatus/race, this paper will cover Coffman and BeLue's findings on race.

Used Item Response Theory (IRT) to ``asses dimensionality''

Null hypothesiss $H_1$, `all item parameters are equal across groups''. If rejected, DIF exists for said SOC (<- is this interpreted right?)
%..............................


%%%% each item has item characteristic curve  Difficulty/Discrimination
\cite{irt:2001}
``parameters of the item characteristic curve''
``items rather than test scores''
Ability is a ``latent trait''
Item -> question on a test


``how well an item can differentiate between
examinees having abilities below the item location and those having abilities
above the item location. This property essentially reflects the steepness of the
item characteristic curve in its middle section. The steeper the curve, the better
the item can discriminate. The flatter the curve, the less the item is able to
discriminate since the probability of correct response at low ability levels is
nearly the same as it is at high ability levels''
%%%.....................

%%%%%
Difficulty is ``value at which probability of endorsing an item is 0.5''~\cite{disparities:2009}.
Difficulty in this experiment is described as the SOC value observed for a question to have a 50\% chance of being agreed with. Figure~%\ref{fig:difficultyGraph} 
represents the values of `difficulty' for each question between blacks and whites and the overall outcome.
Discrimination for each questions was analyzed among all blacks, all whites, and of the overall population. For people who scored as having similar SOC, the y value for each question depicts how often the answer was different among those people.
%%%??????? So (At same SOC, stronger for high values) if high DISC for whites, but low DISC for blacks, that means this question pertains more to a particular race rather than good for determining an accross the board SOC.
Figure~%\ref{fig:discriminationGraph}
depicts these for each question asked. 
%%%...............

%%%
To determine the DIF of a question for determining SOC among groups, an ``information'' value needs to be computed. Baker in~\cite{irt:2001} uses Equation~\ref{eq:variance} to produce information I. $\sigma$ represents the amount of standard deviation, this value is squared to get variance, then the reciprocal is computed to obtain I.
\begin{equation}
\label{eq:variance}
I =  \frac{1}{  \sigma ^{2} } 
\end{equation}
%%%...............

Standard deviation in~\cite{disparities:2009} is described by how much the SOC value varies from a probability of 0 to 1. Therefore, questions will low standard deviations will, based on Equation~\ref{eq:variance}, depict how useful a question is at determining someone's SOC. 


\section{Measures for sense of community and how they differ among race}


\subsection{Measurement items for SOC and determining when they're equivalent}

\section{Differences in importance of community between race}


\section{Community's effect on mental health}
\subsection{Early development}
\subsection{Effect on preceived cognitive learning}
\subsection{Overall effect on mental health}



\section{Conclusion}
Conclusion.

\nocite{*}
%^this is a very important addition so that
%references will be included property

\bibliography{Bibliography}
\bibliographystyle{abbrv}
\end{document}
