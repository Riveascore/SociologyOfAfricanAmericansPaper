\documentclass{sig-alternate}
\usepackage{cite}

\begin{document}

\title{Sense of Community Index, reliability of measurements and their differences among race}
\author{
\alignauthor
Chris Aga \\
\affaddr{ 
University of Minnesota, Morris \\
600 East 4th St. \\
Morris, Minnesota 56267} \\
\email{agaxx010@morris.umn.edu}
}

\date{Today}

\maketitle


\begin{abstract}
Abstract.
\end{abstract}

\keywords{}

\section{Introduction}
Intro.

\section{Sense of Community Background}
Sense of Community (SOC) has various definitions and is extremely important in sociology and psychology. In particular, Sarson introduced SOC in~\cite{sarason:1974} and described it as being one of the most important concepts for definition of self. Sense of Community as defined by McMillan in~\cite{senseOfCommunity:1996} as ``a spirit of belonging together, a feeling that there is an authority structure that can be trusted, an awareness that trade, and mutual benefit come from being together, and a spirit that comes from shared experiences that are preserved as art''. 
\subsection{Operational Constructs}
Varying definitions exist concerning this term, but most literature follows a 4-piece operational system of SOC. The first construct is membership, which describes the idea that when people invest themselves into a community, they have a certain sense of a right to belong~\cite{definition:1986}. Influence, as defined by Peterson as ``a sense that one matters, or can make a difference, in a
community and that the community matters to its members''~\cite{fourFactor:2008}. The third construct, dealing with the idea that the things desired and values shared in a community are similar or are the same. This is termed `reinforcement of needs'~\cite{disparities:2009}. Finally, SOC is also measured in this four-part operational construct system by a `shared emotional connection' between member of the community. 
\paragraph{Validity of four factors}
Many studies have been conducted on these four constructs to determine their relevancy and overall weight in the calculation of an individual's SOC, or Sense of Community Index (SCI). Some have found that these four factors should be measured individually and combined with differing weights to determine one's SCI, while others have found no validity in these measurements and insist that SCI be measured as a one-factor solution. SCI as both a one-factor and four-factor instrument was heavily studied by Chipuer in ~\cite{oneFactor:1999}. The one-factor approach was determined as valid, and the four-factor approach had validity but there was much less consistency of measurements between data-sets than with the one-factor approach.


\section{Experiment}
Coffman in~\cite{disparities:2009} attempted to determine if SCI differences were attributable to true differnces in the amount of SOC between blacks and whites or if any of the questions used to calculate the SCI values were nonequivalent measures between race. This concept is known as differential item functioning, and concerns the prevention of biases of a certain measurement when looking at extrogenous variables such as gender sex or race. 

\subsection{Data}
The dataset used for Coffman's experiment consited of 1,463 interviews to determine SCI. The interviews were conducted ``using computer assisted telephone interview (CATI) technology''~\cite{disparities:2009}. The entire dataset used consisted of 588 whites and 840 blacks and a total of 12 questions were asked to determine an individual's SCI. The questions were yes/no and thus the overall score ranged from 0-12 with higher numbers indicating a stronger perceived SOC. 

\section{Dimensionality}
\subsection{Overview}
Before undergoing the main experiment, Coffman determined the number of factors from the four SOC operational constructs to be used. The final number of factors was carefully determined by a mathematically rigorous process known as full information factor analysis. 

\subsection{Computation and explanation}
Full information item factor analysis first involved calculating the probability of a $person_i$ with a SOC score of $\theta$ answering yes to a particular $question_j$. SOC score in the one-factor model is $\theta$, whereas in for example, a four-factor model (all four operational constructs for sense of community) is represented as scores in: \\$\theta = (membership, influence, similarity, reciprocity)$.\\ Probability is a function of item facility (the number of people who answered yes to a question), standard deviation in answers to a question and the mean answers to a question. 

A probability distribution is generated for the entire data-set and is then integrated over to determine the probability of answering $question_j$ with a certain response. A core factor in the score is using what is known as ``factor loading''. Factor loading is a weight applied to each operational construct for SOC, and these factors are calculated by minimizing the error output of the model. The combination of factor loadings and a person's ability are combined in the probability function as described by Equation~\ref{eq:ability}.
\begin{equation}
\label{eq:ability}
ability = \sum\limits_{k = 1}^{m} \alpha_{jk} \theta_{ki}
\end{equation}

Marginal probability is then computed for $question_j$ based on all the probabilities per person with each specific vector of SOC factor scores. Finally, this marginal probability is combined along with a null hypothesis (not explained at all by the author for this particular experiment) and then a goodness-of-fit test (g-test) is applied. Initially one factor is used, full information analysis is computed, then g-test is applied. If the output is not statistically significant, no further factors are added and the dimensionality is now known for the SCI model~\cite{analysis:1988}.

\section{Choice of parameters logistic model}
\subsection{Models}
After assessing proper dimensionality for the experiment, a proper item response theory model must be chosen to determine which parameters are applicable to each question for studying this data-set. The parameters include difficulty and discrimination. Difficulty in this experiment was the standard deviation from average SOC within a group, of a person having a 50\% chance to answer yes to a question~\cite{disparities:2009}. Discrimination describes 

The second parameter is discrimination, which as defined by Baker in~\cite{irt:2001}, ``describes how well an item can differentiate between examinees having abilities below the item location and those having abilities above the item location.'' Discrimination was calculated using factor loadings which were based off of standard deviation in responses.

Many times a third, `guessing' parameter is used since IRT often is used for knowledge assessment to prevent persons with lesser knowledge from simply achieving a higher score though lucky guessing. This however was not included in Coffman's experiment since most people will not have to resort to guessing to answer questions such as ``I care about what my neighbors think of my actions''.

\subsection{Statistical analysis for final choice}
Three models were suggested and tested to determine the proper approach for this experiment. Two 1PL models, one with a discrimination of 1 for each question, the other with an estimated discrimination value (equivalent for each question), and a 2PL model with discrimination values determined by factors loadings. The model used was chosen by comparing ``Akaike information criterion, and Baeysian information criterion and log-likeliehood nested tests'' for each approach~\cite{disparities:2009}.

\section{Differential Item Functioning}
Once all of the setup is completed, the core component of this experiment can be evaluated. In particular, Coffman wanted to assure that all questions being used to evaluate an individual's SOC were not biased towards a particular group. After parameter value(s) (depending on 1PL or 2PL model) are gathered for each question, differential item functioning (DIF) is applied to each parameter to see if they are variant from group to group. The null hypothesis used for this experiment will be discussed later seeing as they are dependant on the chosen IRT model. Chi-squared 





%....................


\section{Measures for sense of community and how they differ among race}

%%%%%%%%%%%%%%%%%%%%%%

%%%%%%%%%%%%%%%%%%%%%%%




%%%%%%%%%%%%%%%%%%%%%%%
``sense of
community is setting specific.''~\cite{cognitiveLearning:2002}
no universally accepted def
%%%%%%%%%%%%%%%%%%%%%%%

%%%%%%%%%%%%%%%%
Thus SOC is what is known as a latent construct, or as Garger explains in~\cite{latent:2011}, variables that cannot be directly observed ``such as extraversion, intelligence, and self-image''.
%.................

%%%%%%%%%%%%%%%%%%%%%%%
sense of community index~\cite{disparities:2009}
Four dimensions -> membership/influence/reinforcementOfNeeds/sharedEmotionalConnection

Differential Item Functioning to determine accuracy of SOC differences

``DIF among racial groups'' had not been previously studied

DIF applied to -> age/gender/maritialStatus/race, this paper will cover Coffman and BeLue's findings on race.

Used Item Response Theory (IRT) to ``asses dimensionality''

Null hypothesiss $H_1$, `all item parameters are equal across groups''. If rejected, DIF exists for said SOC (<- is this interpreted right?)
%..............................


%%%% each item has item characteristic curve  Difficulty/Discrimination
\cite{irt:2001}
``parameters of the item characteristic curve''
``items rather than test scores''
Ability is a ``latent trait''
Item -> question on a test


``how well an item can differentiate between
examinees having abilities below the item location and those having abilities
above the item location. This property essentially reflects the steepness of the
item characteristic curve in its middle section. The steeper the curve, the better
the item can discriminate. The flatter the curve, the less the item is able to
discriminate since the probability of correct response at low ability levels is
nearly the same as it is at high ability levels''
%%%.....................

%%%%%
Difficulty is ``value at which probability of endorsing an item is 0.5''~\cite{disparities:2009}.
Difficulty in this experiment is described as the SOC value observed for a question to have a 50\% chance of being agreed with. Figure~%\ref{fig:difficultyGraph} 
represents the values of `difficulty' for each question between blacks and whites and the overall outcome.
Discrimination for each questions was analyzed among all blacks, all whites, and of the overall population. For people who scored as having similar SOC, the y value for each question depicts how often the answer was different among those people.
%%%??????? So (At same SOC, stronger for high values) if high DISC for whites, but low DISC for blacks, that means this question pertains more to a particular race rather than good for determining an accross the board SOC.
Figure~%\ref{fig:discriminationGraph}
depicts these for each question asked. 
%%%...............

%%%
To determine the DIF of a question for determining SOC among groups, an ``information'' value needs to be computed. Baker in~\cite{irt:2001} uses Equation~\ref{eq:variance} to produce information I. $\sigma$ represents the amount of standard deviation, this value is squared to get variance, then the reciprocal is computed to obtain I.
\begin{equation}
\label{eq:variance}
I =  \frac{1}{  \sigma ^{2} } 
\end{equation}
%%%...............

Standard deviation in~\cite{disparities:2009} is described by how much the SOC value varies from a probability of 0 to 1. Therefore, questions will low standard deviations will, based on Equation~\ref{eq:variance}, depict how useful a question is at determining someone's SOC. 

If information differences at the same value of SOC are statistically significant then DIF is computed to determine if a $question_x$ is a good indicator at determining SOC differences between race or if $question_x$ is not important for the actual determination of SOC~\cite{disparities:2009}.
Significant differences in a particular question, ``I feel at home in my neighborhood'' were found at areas of similar SOC between blacks and whites in the experiment, thus DIF was calculated on the results to determine how accurate of a determiner this question was for each group. Chi-square 



\section{Differences in importance of community between race}


\section{Community's effect on mental health}
\subsection{Early development}
\subsection{Effect on preceived cognitive learning}
\subsection{Overall effect on mental health}



\section{Conclusion}
Conclusion.

\nocite{*}
%^this is a very important addition so that
%references will be included property

\bibliography{Bibliography}
\bibliographystyle{abbrv}
\end{document}
